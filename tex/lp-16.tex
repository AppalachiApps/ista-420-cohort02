\documentclass{article}
    \usepackage[margin=1in]{geometry}
    \usepackage{hyperref}
    \usepackage{listings}
    \lstset{
        breaklines = true,
        language = SQL,
        numbers = left,
        basicstyle = \footnotesize
    }

    \title{Lesson Plan 16, ISTA-420}
    \author{Chapter 10, T-SQL Fundamentals}
    \date{June 26, 2017}

\begin{document}    

    \maketitle{}

    \section{Class Discussion}

    Pages: 332 --- 348.  No class discussion.

    \section{In Class Exercises}

Do the exercises in ch10-ex.sql in the SQL directory in my Github account.

    \section{Graded Labs}

    LearnOnDemand 10, and if we have time, LearnOnDemand lab 11.

    \section{Course Project}

    Create a new subdirectory in your project directory named \texttt{userdocs}. Document the database portion of your project. You can assume that your readers will be technically proficient in both database technology and your problem domain, baseball.  You will write what is, in essence, a user guide, Your documentation should explain how to use your database. It should also contain a technical reference to your tables, columns, and relations (foreign keys).

    Imagine that you are being interviewed for a job, and your interviewer asks for documentation of a project you have completed. This is the documentation that you will have him look at. If necessary, you can show him your software requirements specification, your entity-relationship diagram, your implementation code, and your testing code, but this user guide should be the final and canonical representation of your project.

    Write this using markdown.  Write this as you would have for your strictest English teacher, paying attention to grammar, punctuation, spelling, and all the other niceties of the English language. When you are judged by non-technical people, this is what they will base their judgment on. In one sense, it's completely irrelevant to your project, as it does not involve requirements analysis, design, iomplementation or testing. In another sense, it is the most visible part of your project, and you should exercise your best efforts to do the best job you can.

    You have two days, Monday and Tiesday, to do your documentation. Place it in your \texttt{userdocs} subdirectory, add it to your git repository, commit it, and push it to your Github account.
    \section{Homework}

            Read chapter  11, pages 361  -- 374  in the \textit{T-SQL Fundamentals} book.
        

\end{document}    
