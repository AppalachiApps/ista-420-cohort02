\documentclass{article}

    \usepackage[margin=1in]{geometry}
    \usepackage{hyperref}
    \usepackage{listings}
    \lstset{
        breaklines = true,
        language = SQL,
        numbers = left,
        basicstyle = \footnotesize
    }

    \title{Lesson Plan 08, ISTA-420}
    \author{Chapter 5, T-SQL Fundamentals}
    \date{June 12, 2017}

\begin{document}    

    \maketitle{}

    \section{Class Discussion}

    Test results, grades, new format of class room exercises, grading rubric for graded exercises and selection of  exercises. 
    Pages 161 -- 183, chapter 5.

    \begin{enumerate}
        \item What is a table expression? Can you give a technical definition of a table expression?
        \item In what SQL clause are derived tables (table valued subqueries) located?
        \item Why can you refer to column aliases in an outer query that you defined in an inner table valued subquery?
        \item What SQL key word defines a common table expression?
        \item When using common table expressions, can a subsequent derived table use a table alias declared in a preceding table expression?
        \item Can a main query refer to a previously defined common table expression by multiple aliases?
        \item In SQL, is a view a durable object?
        \item In a view, what does WITH CHECK OPTION do? Why ius thus important?
        \item In a view, what does SCHEMABINDING do? Why ius thus important?
        \item What is a table valued function?
        \item What does the APPLY operator do?
        \item What are the two forms of the APPLY operator? Give an example of each.
    \end{enumerate}

    \section{In Class Exercises}

    \begin{enumerate}
        \item Write a query that returns all orders placed on the last day of activity that can be found in the Orders table.
        \item Write a query that returns countries where there are customers but not employees. 
        \item Write a query that returns customers who placed orders in 2015 but not in 2016 
        \item Write a query that returns the customer ID,customer name, and country of customers that are not located in the United States using a table valued subquery. 
        \item Do the same using a common table expression.
        \item Do the same using a view.
        \item Write a query returning the last three orders of each customer. Use the APPLY operator.
    \end{enumerate}


    \section{Graded Labs}

    Do LearnOnDemand Lab 5.

    \section{Course Project}

        \subsection{Version Control}

        Markdown: headings, itemized lists, numbered lists.

        \subsection{Submission Instructions}

        Create a directory on your local machine named \texttt{ista420-project}. Initialize a Git repository in that directory. In that directory, create four subdirectories named as follows:

        \begin{verbatim}
\ista420-project
    \analysis
    \design
    \implementation
    \tests
        \end{verbatim}

        Name your two requirements files like this: \texttt{iteration01-requirements.md}. You should have two files, last weeks is \texttt{iteration01} and this weeks will be \texttt{iteration02}. Use Markdown to format the requirements specifications. Place the requirements files in your \texttt{analysis} subdirectory. Add them to your Git repository, commit them, and push them to your Github project repository. Email me the URL of your Github repository at \url{cartec22@erau.edu}.

        \subsection{Project Assignment}

        This week begins the second iteration. Do a second version of your requirements that you developed last week.

    \section{Homework}

        Read pages 193 - 204 in the \textit{T-SQL Fundamentals} book.

\end{document}    
