\documentclass{article}

    \usepackage[margin=1in]{geometry}
    \usepackage{hyperref}
    \usepackage{listings}
    \lstset{
        breaklines = true,
        language = SQL,
        numbers = left,
        basicstyle = \footnotesize
    }

    \title{Lesson Plan 14, ISTA-420}
    \author{Chapter 9, T-SQL Fundamentals}
    \date{June 21, 2017}

\begin{document}    

    \maketitle{}

    \section{Class Discussion}

    Pages: 297 --- 313.

    \begin{enumerate}
        \item What is a temporal table?
        \item Under what circumstances would you use a temporal table? Temporal tables are in widespread use in certain kinds of businesses.
        \item What are the semantics of a temporal table? There are seven of them.
        \item How do you search a history table?
        \item How do you modify a history tablre?
        \item How do you delete date from a history table? Why would you want to delete data from a history table?
        \item How do you search a history table?
        \item How do you query all data from both a history file and the current data?
        \item How do you drop a temporal table?
    \end{enumerate}

    \section{In Class Exercises}

    In the SQL directory, you will find a file named \texttt{lp-14.sql}. It contains six exercises. Load this file into SSMS and work the six exercises.

    \section{Graded Labs}

    Do LearnOnDemand lab no. 9.

    \section{Course Project}

    Implementation. Your deliverable is a SQL file named (more or less) \texttt{iteration4-implementation.sql}.

    \section{Homework}

        Read chapter  10, pages 319  --  322 in the \textit{T-SQL Fundamentals} book.
        

\end{document}    
