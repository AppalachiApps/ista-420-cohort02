\documentclass{article}

    \usepackage[margin=1in]{geometry}
    \usepackage{hyperref}
    \usepackage{listings}
    \lstset{
        breaklines = true,
        language = SQL,
        numbers = left,
        basicstyle = \footnotesize
    }

    \title{Lesson Plan 02, ISTA-420}
    \author{Chapter 2, T-SQL Fundamentals}
    \date{May 31, 2017}

\begin{document}    

    \maketitle{}

    \section{Class Discussion}

            Pages 1 -- 26.

    \begin{enumerate}


        \item Give an informal definition of \textit{database} as used in the expression ``relational database management system.''


        \item Give an informal definition of \textit{database} as used in the expression ``Human Resources database.''


        \item Give an informal definition of \textit{entity integrity}.


        \item Give an informal definition of \textit{referential integrity}.


        \item What is a \textit{relation} as defined in the textbook? A one word answer to this question is sufficient.


        \item Is this table in first normal form? Why or why not? If it is not, how would you change it?

            \begin{verbatim}
create table faculty (
    facID int primary key,
    facName text,
    facCreds text);
            \end{verbatim}

            \begin{tabular}{| l | l | l |}
\hline
\textbf{facID} & \textbf{facName} & \textbf{facCreds} \\
\hline
1 & Alan Alda & BA, MA \\
\hline
2 & Bridgette Bardot & BS, MS, PhD \\
\hline
3 & Casey Cason & AA, BBA, MBA, DEd \\
\hline

            \end{tabular}

        \item Is this table in second normal form? Why or why not? If it is not, how would you change it?

            \begin{verbatim}
create table pets (
    ownerID int primary key,
    petID int primary key,
    ownerName text,
    petName text,
    petType text);
            \end{verbatim}

            \begin{tabular}{| l | l | l | l | l |}
\hline
\textbf{ownerID} & \textbf{petID} & \textbf{ownerName}  & \textbf{petName} & \textbf{petType} \\
\hline
1 & 1 & Dom Delouise & Rex & German Shepherd \\
\hline
1 & 2 & Dom Delouise & Lacy & Border Collie \\
\hline
2 & 3 & Emilio Estevez & Midnight & Persian Cat \\
\hline
            \end{tabular}

        \item Is this table in third normal form? Why or why not? If it is not, how would you change it?

            \begin{verbatim}
create table friends (
    friendID int primary key,
    friendName text,
    friendStreet text,
    friendCity text,
    friendState text,
    friendZip text);
            \end{verbatim}

            \begin{tabular}{| l | l | l | l | l | l | l |}
\hline
\textbf{ID} & \textbf{FirstName} & \textbf{LastName} & \textbf{Street}  & \textbf{City} & \textbf{State} & \textbf{Zip} \\
\hline
1 & Fred & Flintstone & 123 Rock Quarry Rd & Bedrock & GA & 31905 \\  
\hline
2 & Greta & Garbo & 456 Starlit Ave & Paris & FL & 30019 \\  
\hline
3 & Harry & Houdini & 789 Hidden Glen Lane & Alcatraz & CA & 00000 \\  
\hline
            \end{tabular}

        \item What is an \textit{OLTP database}? What operations is it optimized for?


        \item What is a \textit{star schema}? What operations is it optimized for?


    \end{enumerate}


    \section{In Class Exercises}

    \subsection{Installing TSQLV4}

    We will download and install the database the text book uses. We will do this on the machine that you installed SQL Server and SSMS on. You will be using this for the homework.

    \subsection{Class Exercises}

Using SQLite and the Northwind database, write a SQL script that executes the following queries. Your deliverables should be your SQL script and the text output.

        \begin{enumerate}
            \item What are the regions?
            \item What are the cities?
            \item What are the cities in the Southern region?
            \item How do you run this query with the fully qualified column name?
            \item How do you run this query with a table alias?
            \item What is the contact name, telephone number, and city for each customer?
            \item What are the products currently out of stock?
            \item What are the ten products currently in stock with the least amount on hand?
            \item What are the five most expensive products in stock?
            \item How many products does Northwind have? How many customers? How many suppliers?
        \end{enumerate}
        
        \begin{lstlisting}
select * from region;

select TerritoryDescription from territories;

select territorydescription from territories where RegionID = 4;

select territories.territorydescription from territories where RegionID = 4;

select T.territorydescription from territories T where RegionID = 4;

select ContactName, CompanyName, Phone, City from customers;

select productname, unitsinstock from products where unitsinstock = 0;

select productname, unitsinstock from products where unitsinstock > 0 order by unitsinstock limit 10;

select productname, unitprice from products where unitsinstock > 0 order by unitprice desc limit 5;

select count(*) from products;
select count(*) from customers;
select count(*) from suppliers;
        \end{lstlisting}


    \section{Graded Labs}

Lab 1, exercises 1, 2, and 3, from 20761A, page 1-16. Labs will be directed by assigned students.

    \section{Course Project}


        \subsection{Software Process}

        Requirements analysis, functional requirements, nonfunctional requirements. UML and use cases.
        
        \subsection{Version Control}

        Initializing a local Git repository. Setting global configuration parameters.

        \subsection{Project Assignment}

        This project consists of a baseball application. We start with the database, the bottom layer in the layered architecture approach.

        Complete one complete development cycle. You have about an hour to do this. Work in pairs. Your deliverable will be one document described below.

        By necessity, this will be a very simple project, the purpose of which is to give hands-on experience in each phase of the software development process. \textbf{NOTE: the software development and the database development process are two completely different processes.} A database development process consists (in part) of logical database design and physical database design. We will look at this in the coming weeks. For now, I want to focus on the software development process. 

        \begin{enumerate}
            \item \textbf{Requirements analysis:} As a requirements specification, choose one functional requirement (what this application will do) and state it fully. It shold be descriptive enough so that the testing phase can determine whether or not the implementation satisfies the requirement. A good place to begin might be a team roster, but you are encouraged to exercise your creativity. 

            \item \textbf{Design:} State your design. Typically, you would use diagramming software (like Microsoft Visio) to do this, but a freehand drawing is certainly acceptable.  

            \item \textbf{Implementation:} In this case, your implementation will be your SQL script --- the schema of your table. 

            \item \textbf{Testing:} In this case, your tests will be the select queries showing that your implementation meets or fails to meet the requirements specification. 

        \end{enumerate}

    \section{Homework}

        \subsection{Readings}

         Read chapter 2 of the textbook T-SQL Fundamentals, pages 27 through 52.

        \subsection{Exercises}

    \begin{enumerate}

        \item Review the documentation and tutorials on the following web sites.
            \begin{itemize}
                \item Microsoft documentation, \url{https://docs.microsoft.com/en-us/sql/t-sql/queries/queries}
                \item General SQL documentation, \url{https://sqlite.org/syntaxdiagrams.html}
                \item SQL tutorial, \url{https://www.w3schools.com/sql/default.asp}
                \item SQL tutorial, \url{https://www.tutorialspoint.com/t_sql/index.htm}
            \end{itemize}


        \item Do exercises 1 through 5 beginning on page 93. You are allowed to look at and copy the solutions, but make sure you understand the concepts. For example, if you give a query similar to exercise 1 but for another month, would you understand how to do it?

    \end{enumerate}



        


\end{document}    
