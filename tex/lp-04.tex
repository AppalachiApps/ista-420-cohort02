\documentclass{article}

    \usepackage[margin=1in]{geometry}
    \usepackage{hyperref}
    \usepackage{listings}
    \lstset{
        breaklines = true,
        language = SQL,
        numbers = left,
        basicstyle = \footnotesize
    }

    \title{Lesson Plan 04, ISTA-420}
    \author{Chapter 2, T-SQL Fundamentals}
    \date{June 5, 2017}

\begin{document}    

    \maketitle{}

    \section{Class Discussion}

    \begin{enumerate}

        \item 
        \item 
        \item 
        \item 
        \item 
        \item 
        \item 
        \item 
        \item 
        \item 



    \end{enumerate}


    \section{In Class Exercises}

Using SQLite and the Northwind database, write a SQL script that executes the following queries. Your deliverables should be your SQL script and the text output.

    \begin{enumerate}
\item I need a list of our customers and the first name only of the customer representative.
\item You sell some kind of dried fruit that I liked very much. What is its name?
\item Give me a list of our customer contacts alphabetically by last name. 
\item I want to see when customers placed orders in December. Give me a data file showing the day of all December orders.
    \end{enumerate}

    \begin{lstlisting}
select substr(contactname, 1, pos-1) as first_name, companyname
    from (select *, instr(contactname,' ') as pos from customers)
    order by first_name;

select productname from products where productname like "%dried%";;

select strftime("%d", orderdate) as december_day from orders where orderdate like '19__12%';

select --contactname, 
       substr(contactname, instr(contactname, ' ') + 1) || ', ' ||
       substr(contactname, 1, instr(contactname, ' ') - 1) as alphaname
from customers order by alphaname;
    \end{lstlisting}


    \section{Graded Labs}

Lab 2, exercises 1, 2, and 3, from 20761A, page 2-23. Labs will be directed by assigned students.


    \section{Course Project}

        \subsection{Software Process}

        Program design, structured programming, top down development versus bottom up development. 
        
        \subsection{Version Control}

        We will learn how to check the status of a Git repository and to commit files.

        \subsection{Project Assignment}

        This begins your first full week on your project. We will be using an iterative process with iterations one week in length. Typically, on Monday you will do requirements analysis. You will do design on Tuesday. You will do implementation on Wednesday. You will do testing on Thursday.

        Since today is Monday, you will do requirements analysis. Focus on functional requirements, that is, what the application is supposed to do. You have been employed to build an application supporting a baseball team. The foundation for all applications is the data layer. (What would your application do if it had absolutely no data?) Therefore, you will start with the design of a database. Start very simple. You will increase the complexity as the term progresses. Your assignment for today is to draft a requirements specification for the application. \textit{Do not go beyond completing the requirements specification --- no design, implementation, or testing!} Your deliverable will be your requirements specification.


    \section{Homework}


        \subsection{Readings}
        Read pages 73 -- 102 in the \textit{T-SQL Fundamentals} book.

        
        \subsection{Exercises}

        \begin{itemize}
        \item Do exercises 6 through 10 beginning on page 93. You are allowed to look at and copy the solutions, but make sure you understand the concepts. For example, if you give a query similar to exercise 1 but for another month, would you understand how to do it?

        \item Read the Microsoft T-SQL documentation for the functions used in these exercises.
        \end{itemize}
\end{document}    
