\documentclass{article}

    \usepackage[margin=1in]{geometry}
    \usepackage{hyperref}
    \usepackage{listings}
    \lstset{
        breaklines = true,
        language = SQL,
        numbers = left,
        basicstyle = \footnotesize
    }

    \title{Lesson Plan 13, ISTA-420}
    \author{Chapter 8, T-SQL Fundamentals}
    \date{June 20, 2017}

\begin{document}    

    \maketitle{}

    \section{Class Discussion}

    Pages: 266 --- 287

    \begin{enumerate}
        \item 
        \item 
        \item 
        \item 
        \item 
        \item 
        \item 
        \item 
        \item 
        \item 
    \end{enumerate}

    \section{In Class Exercises}


    \begin{enumerate}
\item Look at the csv file named presidents-short.csv in the SQL directory. Create an appropriate table schema.

\item Insert the CSV data into the table you just created.

\item Delete the first record from your table using the output clause. This is the header.

\item Alter the presidents table by adding a primary key column .

\item Bring the data up to date by updating the last row and adding a new row. Use the output clause.

\item Alter the presidents table to update the current party names, use the output clause.  Federalists and Whigs became Republicans.  Democratic-Republicans became Democrats
    \end{enumerate}

    \section{Graded Labs}

    LearnOnDemand lab 8.

    \section{Course Project}

    Iteration 3 design phase.

    \section{Homework}

        Read chapter 9, pages 297  -- 313  in the \textit{T-SQL Fundamentals} book.
        

\end{document}    
