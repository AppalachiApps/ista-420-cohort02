\documentclass{article}

    \usepackage[margin=1in]{geometry}
    \usepackage{hyperref}
    \usepackage{listings}
    \lstset{
        breaklines = true,
        language = SQL,
        numbers = left,
        basicstyle = \footnotesize
    }

    \title{Lesson Plan 06, ISTA-420}
    \author{Chapter 3, T-SQL Fundamentals}
    \date{June 7, 2017}

\begin{document}    

    \maketitle{}

    \section{Class Discussion}

    Pages 103 -- 123.

    \begin{enumerate}
        \item In general, why would you even want to join two (or more) tables together? This is a good time to think about the nature of relational algebra.
        \item Describe in your own words the output from an \textit{inner join}.
        \item Describe in your own words the output from an \textit{outer join}.
        \item Describe in your own words the output from an \textit{cross join}.
        \item A convenient mnemonic for remembering the various joins is ``Ohio.'' Why is this true?
        \item Give an example of a \textit{composite join.}
        \item What is the difference between the following two queries? The business problem is ``How many orders do we have from each customer?''

        \begin{verbatim}
        ================first query===============
SELECT C.custid, COUNT(*) AS numorders
FROM Sales.Customers AS C
LEFT OUTER JOIN Sales.Orders AS O
ON C.custid = O.custid
GROUP BY C.custid;
        ================second query===============
SELECT C.custid, COUNT(O.orderid) AS numorders
FROM Sales.Customers AS C
LEFT OUTER JOIN Sales.Orders AS O
ON C.custid = O.custid
GROUP BY C.custid;
        \end{verbatim}
    \item What might be one reason the following query does not return the column \textit{custID} in this query?

        \begin{verbatim}
SELECT C.custid, C.companyname, O.orderid, O.orderdate
FROM Sales.Customers AS C
LEFT OUTER JOIN Sales.Orders AS O
ON C.custid = O.custid
WHERE O.orderdate >= '20160101';
        \end{verbatim}

    \end{enumerate}

    \section{In Class Exercises}
Using SQLite and the Northwind database, write a SQL script that executes the following queries. Your deliverables should be your SQL script and the text output.

        \begin{enumerate}
            \item What is the order number and the date of each order sold by each employee?
            \item List each territory by region.
            \item What is the supplier name for each product alphabetically by supplier?
            \item For every order on May 5, 1998, how many of each item was ordered, and what was the price of the item?
            \item For every order on May 5, 1998, how many of each item was ordered giving the name of the item, and what was the price of the item?
            \item For every order in May, 1998, what was the customer's name and the shipper's name?
            \item What is the customer's name and the employee's name for every order shipped to France?
            \item List the products by name that were shipped to Germany.
        \end{enumerate}

        \begin{lstlisting}
select e.employeeid, e.firstname, e.lastname, o.orderid, o.orderdate from employees e join orders o on e.employeeid = o.employeeid;
select e.employeeid, e.firstname, e.lastname, o.orderid, o.orderdate from employees e, orders o where e.employeeid = o.employeeid;

select r.regiondescription, t.territorydescription from territories t join region r on r.regionid = t.regionid;
select r.regiondescription, t.territorydescription from territories t, region r where r.regionid = t.regionid;

select p.productname, s.companyname from products p join suppliers s on s.supplierid = p.supplierid order by s.companyname;
select p.productname, s.companyname from products p, suppliers s where s.supplierid = p.supplierid order by s.companyname;

select o.orderdate, o.orderid, d.productid, d.quantity, d.unitprice from order_details d join orders o on o.orderid = d.orderid where o.orderdate = '1998-05-05';
select o.orderdate, o.orderid, d.productid, d.quantity, d.unitprice from order_details d, orders o where o.orderid = d.orderid and o.orderdate = '1998-05-05';

select o.orderdate, o.orderid, p.productname, d.quantity, d.unitprice from order_details d join orders o on o.orderid = d.orderid join products p on p.productid = d.productid where o.orderdate = '1998-05-05';
select o.orderdate, o.orderid, p.productname, d.quantity, d.unitprice from order_details d, orders o, products p where o.orderid = d.orderid and p.productid = d.productid and o.orderdate = '1998-05-05';

select o.orderid, o.orderdate, c.companyname, s.companyname from orders o join customers c on o.customerid = c.customerid join shippers s on s.shipperid = o.shipperid where o.orderdate like '1998-01%';
select o.orderid, o.orderdate, c.companyname, s.companyname from orders o, customers c, shippers s where o.customerid = c.customerid and s.shipperid = o.shipperid and o.orderdate like '1998-01%';

select o.orderid, c.companyname, e.firstname, e.lastname, o.shipcountryfrom orders o join customers c on o.customerid = c.customerid join employees e on o.employeeid = e.employeeid where o.shipcountry = 'France';
select o.orderid, c.companyname, e.firstname, e.lastname, o.shipcountry from orders o, customers c, employees e where o.customerid = c.customerid and  o.employeeid = e.employeeid and o.shipcountry = 'France';

select distinct p.productname, o.shipcountry from orders o join order_details d on o.orderid = d.orderid join products p on  d.productid = p.productid where o.shipcountry = 'Germany'; 
select distinct p.productname, o.shipcountry from products p, orders o, order_details d where o.orderid = d.orderid and d.productid = p.productid and o.shipcountry = 'Germany'; 
        \end{lstlisting}


    \section{Graded Labs}



    \section{Course Project}

        \subsection{Software Process}

        Implementation. Structured programming. Top down versus bottom up development. DSLs.
        
        \subsection{Version Control}

        Review of Git and Github: init, add, status, commit, push.

        \subsection{Project Assignment}

        Implement your database design you worked on yesterday. Your deliverable will be your SQL script that you write implementing the database.

    \section{Homework}


        \subsection{Readings}

        Read pages 133 -- 149 in the \textit{T-SQL Fundamentals} book.
        
        \subsection{Exercises}

        Do all exercises in chapter 3.
\end{document}    
