\documentclass{article}

    \usepackage[margin=1in]{geometry}
    \usepackage{hyperref}
    \usepackage{listings}
    \lstset{
        breaklines = true,
        language = SQL,
        numbers = left,
        basicstyle = \footnotesize
    }

    \title{Lesson Plan 06, ISTA-420}
    \author{Chapter 4, T-SQL Fundamentals}
    \date{June 7, 2017}

\begin{document}    

    \maketitle{}

    \section{Class Discussion}

    Pages 103 -- 123.

    \begin{enumerate}
        \item In your own words, what is a \textit{subquery}?
        \item What is the difference between a subquery that returns a single value and one that returns multiple values? Give examples of where you would use each kind of subquery. 
        \item 
        \item 
        \item 
        \item 
        \item 
        \item 
        \item 
        \item 
    \end{enumerate}

    \section{In Class Exercises}

    \begin{enumerate}
        \item What products do we have that cost more than the average price of all of our products?
        \item What is our highest priced product? Use a subquery for the answer.
        \item 
        \item 
        \item 
        \item 
        \item 
        item 
        \item 
        \item 
    \end{enumerate}

    \section{Graded Labs}



    \section{Course Project}


        \subsection{Software Process}

        
        \subsection{Version Control}


        \subsection{Project Assignment}

    \section{Homework}


        \subsection{Readings}

        Read pages 133 -- 149 in the \textit{T-SQL Fundamentals} book.
        
        \subsection{Exercises}

        Do exercises 1 -- 5 in chapter 4.
\end{document}    
