\documentclass{article}

    \usepackage[margin=1in]{geometry}
    \usepackage{hyperref}
    \usepackage{listings}
    \lstset{
        breaklines = true,
        language = SQL,
        numbers = left,
        basicstyle = \footnotesize
    }

    \title{Lesson Plan 09, ISTA-420}
    \author{Chapter 6, T-SQL Fundamentals}
    \date{June 13, 2017}

\begin{document}    

    \maketitle{}

    \section{Class Discussion}

    Pages 193 -- 204, chapter 6.

    \begin{enumerate}
        \item What does a set operator do?
        \item What are the general requirements of a set operator
        \item What is a Venn Diagram? This is not in the book.
        \item Draw a Venn Diagram of the UNION operator. What does it do?
        \item Draw a Venn Diagram of the UNION ALL operator. What does it do?
        \item Draw a Venn Diagram of the INTERSECT operator. What does it do?
        \item If SQL Server supported the INTERSECT ALL operator, what would it do?
        \item Draw a Venn Diagram of the EXCEPT operator. What does it do?
        \item If SQL Server supported the EXCEPT ALL operator, what would it do?
        \item What is the precedence of the set operators?
    \end{enumerate}

    \section{In Class Exercises}

    \begin{enumerate}
        \item List the cities, regiions, and countries in which we have a supplier or a customer. 
        \item List the cities, regiions, and countries in which we have a supplier or a customer, with duplicates.
        \item List only the cities, regions, and countries in which we have both a supplier and a customer. 
        \item List the cities, regions, and countries in which we have a supplier but not a customer. 
        \item List the cities, regions, and countries in which we have a customer but not a supplier. 
    \end{enumerate}

    \section{Graded Labs}

    Do LearnOnDemand Lab 6.

    \section{Course Project}

        \subsection{Software Process}

        Database design. Logical design and physical design.
        
        \subsection{Project Assignment}

        Do an ERD (logical design) to implement your iteration 2 requirements. Upload your drawing to Github in the appropriate directory.

        If you have time, try your hand at a physical design, using Visio or another drawing tool.

    \section{Homework}

        Read pages 213 - 230 in the \textit{T-SQL Fundmentals} book.
        
\end{document}    
